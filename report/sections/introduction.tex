\section{Introdução}
\label{sec:intro}

No escopo de disciplina de Organização e Arquitetura de Computadores I o
terceiro trabalho pode ser resumido da seguinte maneira: dado quatro problemas,
estes devem ser resolvidos em uma linguagem de alto nível, no caso deste
trabalho foi escolhido python, e também deve ser apresetnada a solução
utilizando a linguagem assembly do processador cleópatra, como estudado em aula.

Os problemas a serem resolvidos são os seguintes:

\begin{itemize}

    \item Descobrir se um número \emph{n} positivo é múltiplo de número do grupo
        +5(se por acaso, o valor de grupo + 5 ultrapassar o número do maior
        grupo, deve ser subtraído o número do maior grupo.

    \item Fazer um programa que gere os \emph{n} primeiros números da sequência
        de Fibonnacci, e armazene a sequências em endereços consecutivos a
        partir da posição de memória com rótulo idx. Esta operação deve ser
        feita com uma chamada de função para o rótulo SeqFibonnacc i.  

    \item Faça um programa que calcule a multiplicação de todos os elementos da
        diagonal principal de uma matriz 5x5. Obs.: o algoritmo deve ser
        genérico para qualquer matriz quadrada. Este programa deve chamar
        uma função para fazer a multiplicação


    \item Faça um programa que contenha a função Potencia. Esta função deve
        ser capaz de receber dois valores \emph{V} e \emph{EXP} e ter como
        saída $V{exp}$. Esta saída deve estar associada ao acumulador


\end{itemize}
